\chapter*{Abstract}
\addcontentsline{toc}{chapter}{Abstract}
In the field of computational fluid dynamics (CFD), the most popular
method to resolve the effects of turbulence in a numerical simulation is to
employ the Reynolds-averaged Navier-Stokes Equations (RANS) and
capture the turbulence effects using a turbulence model. Among the many
turbulence models that exist, the Spalart-Allmaras (S-A) model has proven to
be reliable for attached and moderately separated flows. In this work, the S-A model has been implemented in two CFD frameworks: an in-house academic
structured finite-volume cell-centered code and MAYA HTT’s commercial
unstructured vertex-centered code. The implementation details are presented along with validation results on NASA Turbulence Model Resource benchmark cases and on aerodynamic
cases.


\selectlanguage{french}
\chapter*{Abrégé}
\addcontentsline{toc}{chapter}{Abrégé}
Dans le domaine de la mécanique des fluides numérique (MFN), l'approche la plus populaire pour prédire le comportement d'un écoulement turbulent est de faire usage à une méthode de moyennage temporel des équations de Navier-Stokes. Les effets turbulents sont donc pris en compte par un modèle de turbulence. Dans le cadre de cette thèse, le modèle conçu par Spalart et Allmaras, lequel est reconnu pour sa fiabilité pour les écoulements externes, est implémenté dans deux logiciels MFN, l'un étant académique et l'autre étant commercial. Les détails de l'implémentation ainsi que des résultats de validation sont présentés. 
\selectlanguage{english}