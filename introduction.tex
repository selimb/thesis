\chapter{Introduction}
This work was done in collaboration with Maya Heat Transfer Technologies, a company developing its own flow solver named NX Flow, part of the Siemens PLM portfolio. The aim of this thesis is the implementation and validation of the Spalart-Allmaras turbulence model in NX Flow as well as the in-house academic code at McGill's Computational Aerodynamics Group, named syn3D.


\section{Computational fluid dynamics}
Computational fluid dynamics (CFD) is a branch of fluid mechanics that uses numerical analysis to solve problems that involve fluid flows. The fundamental basis to all CFD problems is the Navier-Stokes equations, a set of mathematical equations governing the dynamics of any fluid flow. Surprisingly, numerical methods for CFD were first developed in the 1910's by Lewis Fry Richardson even prior to the first computers and were carried out by hand. Then, a team at the Los Alamos National Lab in 1957 developed the first functional CFD computer simulation model. Due to a lack of computational resources, only a simplified set of equations could be solved. 

As time went on and computers became more powerful and the full Navier-Stokes could finally be solved. The theory was there, it was merely computational power that was lagging behind. In fact, a lot of methods used in this work still date back to the 1970's!
\section{Turbulence modelling}
The effects of turbulence on fluid flow, characterized by its chaotic nature, still pose a challenge in CFD to this day, even though the study of turbulence dates back to as far as Leonardo Da Vinci. 

It is extremely costly to numerically resolve turbulent effects, which makes it impossible to perform simulations on real-life engineering cases. At the expense of accuracy, it is possible to instead model these effects. A common approach is to employ an averaging technique called Reynolds averaging and solve the Reynolds-averaged Navier-Stokes equations (RANS), which yield a solution for the mean flow. Another approach, large eddy simulation (LES), which is more costly but typically yields a more accurate solution than RANS, resolves most of the turbulence effects while modelling the smaller scales. 

Solving the RANS equations requires evaluation of the eddy viscosity, the quantity that models the effects of turbulence. This quantity is solved for through a turbulence model. One such turbulence model is called the Spalart-Allmaras model, and is the focus of this work. This particular model was designed for aerospace applications and has been shown to give good results for boundary layers subject to adverse pressure gradients~\cite{spalart1994one}. It also involves solving only one additional transport equation, as opposed to other turbulence models like $k-\epsilon$ and $k-\omega$ which require solving two equations. Thus, the Spalart-Allmaras model offers a good trade-off between computational cost and solution accuracy and is well suited for the cases discussed in this work. Another reason for the choice of this model is that a customer of NX Flow requested it.
\section{Thesis overview}
The governing equations and an introduction to turbulence modelling are presented in~\Cref{chap:governing}. Numerical methods and their implementation in both solvers are given in~\Cref{chap:num}. Finally, results are presented in~\Cref{chap:results}. 