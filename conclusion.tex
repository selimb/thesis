\chapter{Conclusion}
This chapter concludes the work. The obtained results are discussed in~\Cref{sec:conclusionresults} and future work is given in~\Cref{sec:future}.

\section{Discussion of results}
\label{sec:conclusionresults}
The results following the implementation of the Spalart-Allmaras turbulence model in NX Flow and syn3D were compared to established CFD codes as well as experimental data. Both codes were run on compressible and essentially incompressible cases. All cases require Reynolds numbers between 2 million and 9 million. The results can only be as accurate as the rest of the existing solver, specifically the implementation and discretization of Navier-Stokes equations as well as its boundary conditions, since the eddy viscosity obtained from the S-A model both feeds into these equations and also relies on the solution of the flow variables.

Both syn3D and NX Flow gave excellent results for the flat plate problem. However, it was found that syn3D predicts a spike in the eddy viscosity at the interface between symmetry BCs and wall BCs. This effect is more pronounced for coarser grids, which may explain why syn3D exhibits greater sensitivy to grid density as compared to other codes, including NX Flow. Another factor which contributes to this sensitivity is that the artificial dissipation parameters were not tuned for each individual mesh level. Additionally, it was seen that there exists small differences in the drag value between compressible codes and codes using a fixed density, even if that problem is run at essentially incompressible conditions ($M = 0.2$). Finally, it was seen that the temporal discretization significantly affect the residual convergence: whereas the fully-implicit backward Euler method used by NX Flow resulted in expensive but effective iterations, the explicit modified Runge-Kutta scheme employed by syn3D required around a thousand times more iterations, albeit the cost of each iteration is greatly reduced. However, ease of implementation and memory requirements are important factors to consider, both of which favor the explicit approach.

Syn3D was also validated on the two-dimensional and three-dimensional bump cases from the Turbulence Modeling Resource. The spike in eddy viscosity, similar to that observed for the flat plate, was made even more apparent on these problems. It was also found that the coefficient of pressure is less sensitive to grid spacing than the skin friction coefficient. Moreover, convergence of the residual was shown to be slow on these cases. Increasing the number of multigrid levels might accelerate convergence, but optimization of convergence acceleration methods is not the focus of this work.

NX Flow was then validated on an essentially incompressible NACA0012 case at various angles of attack. While the pressure coefficient results were on par with CFL3D for the most part, the integrated drag value was significantly over-predicted due to an inaccurate prediction of the flow acceleration at the leading edge on the upper surface of he airfoil. This discrepancy is due to an unsuitable BC for the far-field boundaries and a grid that may be too coarse considering the weak treatment of the solid wall BC.

Both NX Flow and Syn3D were compared against experimental data for the RAE2822 airfoil, this time in compressible flow. Both codes deviated slightly from the expected results, although NX Flow exhibited the same problems as with the NACA0012. Syn3D was able to accurately predict the location of the shock for the transonic case.

Preliminary results were obtained for a high-lift configuration with syn3D, which are qualitatively correct. However, the computational grid is still much too coarse; improvements to the discretization as well as tuning of the parameters may be necessary in order to run a finer grid.

To conclude, it can be said that the implementation in both codes corresponds to the discretization, i.e. no programming errors have been made. This doesn't mean, however, that improvements cannot be made, either to the underlying solver or to the SA implementation itself.

\section{Future work}
\label{sec:future}
To further validate, several items are suggested.
First, what seems to be a boundary condition issue in syn3D, which led to high peaks in the eddy viscosity for the TMR cases, should be investigated. This may decrease the dependence on grid refinement.

Second, a grid study should be performed on the NACA0012 and RAE2822 to see if better results can be obtained.

Third, the effects of strong and weak boundary conditions for viscous walls in NX Flow should be compared.

Fourth, a far-field BC should be implemented into NX Flow, so as to be able to run external flow simulations on structured grids, which are popular for such flows. For best results, this BC should employ a vortex correction in order to reduce influence of the far-field condition on the solution and also reduce dependence on the extent of the computational grid~\cite{tmrvortex,thomas1986far}.

Finally, the high-lift configuration should be solved on a suitable grid with syn3D, along with a transition model.
